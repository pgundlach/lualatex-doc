\documentclass{lltxdoc}

\title{A guide to \lualatex}
\author{Manuel Pégourié-Gonnard \\ \email{mpg@elzevir.fr}}
\date{\today}

\begin{document}

\maketitle

\begin{abstract}
  This document guides the reader into the new world of
  \lualatex.\footnote{Though focusing on \lualatex, it includes relevant
    information about \luatex with the Plain format, too.} The intended
  audience ranges from complete newcomers (with a working knowledge of
  conventional \latex, though) to package developers. This guide is
  intended to be comprehensive in the following sense: it contains pointers to
  all relevant sources, gathers information that is otherwise scattered,
  and adds introductory material.

  Feedback, especially on the introduction, is most welcome.\footnote{The
    document is currently at an early stage of writing.  Thanks for your
    comprehension and patience.}
\end{abstract}

\setcounter{tocdepth}{2}
\tableofcontents
\clearpage

\section{Introduction}

Bla \verb+\relax+ bla.

\subsection{Just what is \lualatex?}

\subsection{Switching from \latex to \lualatex for the impatient}

\subsection{A Lua-in-\tex primer}

\subsection{Other things you should know}

\section{Essential packages and practices}

\subsection{User-level}

\subsubsection{Fontspec}

\subsection{Developer-level}

\subsubsection{Engine detection: ifluatex, iftex, expl3}

\subsubsection{Mode detection: ifpdf}

\subsubsection{Luatexbase}

\subsubsection{Luatex}

\subsubsection{Font internals: luaotfload, euenc, xunicode}

\subsubsection{Lualibs}

\section{Other packages}

\subsection{User-level}

\subsubsection{Luainputenc}

\subsubsection{Luamplib}

\subsubsection{Microtype}

\subsubsection{Luacolor}

\subsubsection{Luadirections}

\subsection{Developer-level}

\subsubsection{Pdftexcmds}

\subsubsection{Magicnum}

\subsubsection{Lua-alt-getopt}

\section{The \luatex\ and \lualatex\ formats}

\end{document}

% vim: set spell spelllang=en
